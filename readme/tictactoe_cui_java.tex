% latex uft-8
\documentclass[uplatex,a4paper,11pt,oneside,openany]{jsbook}
%
\usepackage[dvipdfmx]{graphicx}
\usepackage{amsmath,amssymb}
\usepackage{bm}
\usepackage{graphicx}
\usepackage{ascmac}
\usepackage{setspace}
\usepackage{here}
\usepackage{listings,jlisting} %日本語のコメントアウトをする場合jlistingが必要
%ここからソースコードの表示に関する設定
\usepackage{xcolor,comment}

\definecolor{mygreen}{rgb}{0,0.6,0}
\definecolor{mygray}{rgb}{0.5,0.5,0.5}
\definecolor{mymauve}{rgb}{0.58,0,0.82}

\begin{comment}
\lstset{
  backgroundcolor=\color{white},   % choose the background color; you must add \usepackage{color} or \usepackage{xcolor}; should come as last argument
  basicstyle=\footnotesize,        % the size of the fonts that are used for the code
  breakatwhitespace=false,         % sets if automatic breaks should only happen at whitespace
  breaklines=true,                 % sets automatic line breaking
  captionpos=b,                    % sets the caption-position to bottom
  commentstyle=\color{mygreen},    % comment style
  deletekeywords={...},            % if you want to delete keywords from the given language
  escapeinside={\%*}{*)},          % if you want to add LaTeX within your code
  extendedchars=true,              % lets you use non-ASCII characters; for 8-bits encodings only, does not work with UTF-8
  firstnumber=1000,                % start line enumeration with line 1000
  frame=single,	                   % adds a frame around the code
  keepspaces=true,                 % keeps spaces in text, useful for keeping indentation of code (possibly needs columns=flexible)
  keywordstyle=\color{blue},       % keyword style
  language=Octave,                 % the language of the code
  morekeywords={*,...},            % if you want to add more keywords to the set
  numbers=left,                    % where to put the line-numbers; possible values are (none, left, right)
  numbersep=5pt,                   % how far the line-numbers are from the code
  numberstyle=\tiny\color{mygray}, % the style that is used for the line-numbers
  rulecolor=\color{black},         % if not set, the frame-color may be changed on line-breaks within not-black text (e.g. comments (green here))
  showspaces=false,                % show spaces everywhere adding particular underscores; it overrides 'showstringspaces'
  showstringspaces=false,          % underline spaces within strings only
  showtabs=false,                  % show tabs within strings adding particular underscores
  stepnumber=2,                    % the step between two line-numbers. If it's 1, each line will be numbered
  stringstyle=\color{mymauve},     % string literal style
  tabsize=2,	                   % sets default tabsize to 2 spaces
  title=\lstname                   % show the filename of files included with \lstinputlisting; also try caption instead of title
}
\end{comment}

%ここからソースコードの表示に関する設定

\lstdefinestyle{customc}{
  belowcaptionskip=1\baselineskip,
  breaklines=true,
  numbers=left,
  frame=single,
  xleftmargin=\parindent,
  language=C,
  showstringspaces=false,
  basicstyle=\footnotesize\ttfamily,
  keywordstyle=\bfseries\color{green!40!black},
  commentstyle=\itshape\color{purple!40!black},
  identifierstyle=\color{blue},
  stringstyle=\color{orange},
}

\lstdefinestyle{customjava}{
  belowcaptionskip=1\baselineskip,
  breaklines=true,
  numbers=left,
  frame=single,
  xleftmargin=\parindent,
  language=java,
  showstringspaces=false,
  basicstyle=\footnotesize\ttfamily,
  keywordstyle=\bfseries\color{green!40!black},
  commentstyle=\itshape\color{purple!40!black},
  identifierstyle=\color{blue},
  stringstyle=\color{orange},
}

\lstdefinestyle{customasm}{
  belowcaptionskip=1\baselineskip,
  frame=L,
  xleftmargin=\parindent,
  language=[x86masm]Assembler,
  basicstyle=\footnotesize\ttfamily,
  commentstyle=\itshape\color{purple!40!black},
}

\lstset{escapechar=@,style=customjava}

\makeatletter
\def\ps@plainfoot{%
  \let\@mkboth\@gobbletwo
  \let\@oddhead\@empty
  \def\@oddfoot{\normalfont\hfil-- \thepage\ --\hfil}%
  \let\@evenhead\@empty
  \let\@evenfoot\@oddfoot}
  \let\ps@plain\ps@plainfoot
\renewcommand{\chapter}{%
  \if@openright\cleardoublepage\else\clearpage\fi
  \global\@topnum\z@
  \secdef\@chapter\@schapter}
\makeatother
%
\newcommand{\maru}[1]{{\ooalign{%
\hfil\hbox{$\bigcirc$}\hfil\crcr%
\hfil\hbox{#1}\hfil}}}
%
\setlength{\textwidth}{\fullwidth}
\setlength{\textheight}{40\baselineskip}
\addtolength{\textheight}{\topskip}
\setlength{\voffset}{-0.55in}
%
\begin{document}
% START DOCUMENT
%
% COVER
\begin{center}
  \huge \par
  \vspace{15mm}
  \huge \par
  \vspace{15mm}
  \LARGE Tic Tac Toe (CUI - java) \par
  \vspace{100mm}
  \Large \date \par
  \vspace{15mm}
  \Large \par
  \vspace{10mm}
  \Large S.Matoike \par
  \vspace{10mm}
\end{center}
\thispagestyle{empty}
\clearpage
\addtocounter{page}{-1}
\newpage
\setcounter{tocdepth}{3}
%
\tableofcontents
%
\chapter{javaによる三目並べ [Tic Tac Toe]}

\section{制作する三目並べの概要}

プログラムを実行すると、盤面が表示され、×の石を置く場所を指定するよう促されます

画面上に示された番号を入力すると、その番号のスロットに×の石が置かれた盤面が表示され、次の手番の○に、石を置く場所を指定するように促されます

手番を交互に変えながらゲームは進み、縦、横、斜めの何れかに、先に一列に自分の石を並べた方が勝ちとなります

既に石の置かれているスロット番号を指定できませんし、スロット番号として 0 〜 8 以外の数値を指定することもできません

\begin{spacing}{0.74}
  \begin{verbatim}
    スタート! [Tic Tac Toe]
    /---|---|---\
    | 0 | 1 | 2 |
    |---|---|---|
    | 3 | 4 | 5 |
    |---|---|---|
    | 6 | 7 | 8 |
    \---|---|---/
    'X' さんのturnです
    石を置く場所 0 〜 8 を指定して下さい : 4
    /---|---|---\
    | 0 | 1 | 2 |
    |---|---|---|
    | 3 | X | 5 |
    |---|---|---|
    | 6 | 7 | 8 |
    \---|---|---/
    'O' さんのturnです
    石を置く場所 0 〜 8 を指定して下さい : 2
    /---|---|---\
    | 0 | 1 | O |
    |---|---|---|
    | 3 | X | 5 |
    |---|---|---|
    | 6 | 7 | 8 |
    \---|---|---/
    'X' さんのturnです
    石を置く場所 0 〜 8 を指定して下さい :
  \end{verbatim}
\end{spacing}

\section{定数定義のクラスと主処理}

Constantsクラスに定数の一式をおいて、他のクラスと共有するようにします

\lstinputlisting[caption=定数定義:Tic Tac Toe,label=prog1]{../src/tictactoe/Constants.java}

コマンドライン引数の指定で、先手と後手を入れ替えられるようにしています

\lstinputlisting[caption=main関数:Tic Tac Toe,label=prog2]{../src/tictactoe/TicTacToe.java}

\section{ゲームクラス}

コンストラクタで、2人のプレイヤー、1つの盤面の各インスタンスを用意します

\lstinputlisting[caption=Gameクラス:Tic Tac Toe,label=prog3]{../src/p3.src}

①盤面を表示し、②現在手番のプレーヤーに打つ手を決めさせ、③それを盤面に置きます その次に④盤面の状態から勝敗を判定し、⑤手番を交代すること、これを勝敗がつくまで繰り返しています\\

勝敗の確認、board.check() の結果を引数として受け取り、それに基づいて結果を表示します

(上記のプログラム中で、//結果を表示する関数、の部分に収まる関数です)

\lstinputlisting[caption=結果を表示する関数:Tic Tac Toe,label=prog3_1]{../src/p3_1.src}

\section{石のクラス}

石のクラスが持っているのは、その石が置かれる盤面上の位置(locate)と、そこへ置く石の種類(id=MARU or BATSU)です

private な変数には、public な getter と setter を通してアクセスさせます

\lstinputlisting[caption=石クラス:Tic Tac Toe,label=prog4]{../src/tictactoe/Stone.java}

\section{盤面のクラス}

盤面クラスが持っているのは、9つの石を置く場所を持つゲーム盤 stones[0] 〜 stones[8] です

盤面の初期状態は、石を置く場所の全てが EMPTY の状態で、コンストラクタがその設定を行います

canPut( Stone ) は、引数で受け取った石が、現在の盤面に置けるのかどうかをチェックします

setBoard( Stone ) は、引数で受け取った石を、盤面上の指定された位置に置きます

\lstinputlisting[caption=盤面クラス:Tic Tac Toe,label=prog5]{../src/p5.src}

盤面上の石を置く場所には '0' 〜 '8' の数字を書き、MARU や BATSU の石の置かれた場所には、'O' あるいは 'X' の記号を書いています\\

(以下は、//盤面を評価する関数、の部分に収まる関数です)

\lstinputlisting[caption=盤面を表示する関数:Tic Tac Toe,label=prog5_1]{../src/p5_1.src}

勝敗を判定するには、横3行、縦3列、あるいは2つの斜めのライン上に、MARU あるいは BATSU が並んでいるかどうかを調べます

盤面上に EMPTY の状態の場所があったら、まだゲームは進行中(NEXT)です

盤面上に EMPTY の状態の場所が無くなっており、かつ勝敗がついていないのなら、それは引き分け(DRAW)です\\

(以下は、//勝敗を判定する関数、の部分に収まる関数です)

\lstinputlisting[caption=勝敗を判定する関数:Tic Tac Toe,label=prog5_2]{../src/p5_2.src}

(以下は、//もう一つの勝敗を判定する関数、の部分に収まる関数です)

\lstinputlisting[caption=勝敗を判定する関数その2:Tic Tac Toe,label=prog5_3]{../src/p5_3.src}

(以下は、後述する戦略MiniMax法で使う関数です)

\lstinputlisting[caption=局面を評価する関数:Tic Tac Toe,label=prog5_4]{../src/p5_4.src}

\section{プレイヤーのクラス}

プレイヤークラスが持っているのは、プレイヤーの氏名、そのプレイヤーが採ろうとしている戦略、そのプレイヤーが置こうとしている石の種類(MARU か BATSU か)です

putStone( Board ) の処理では、\\プレーヤーが選んでいる戦略( Senryaku )を使って決めた手 te=Te()(=石の置き場所)を、\\引数で受け取ったその時の盤面( board )に置けるのかどうかを確認( canPut( Stone ) )して、\\置けるならば、盤面にその石を置きます( board.setBoard( Stone ) )

この Playerクラスは、戦略クラス(class Strategy)を継承しています

\lstinputlisting[caption=プレイヤークラス:Tic Tac Toe,label=prog6]{../src/tictactoe/Player.java}

\section{戦略クラス(MiniMax)}

Player クラスの親クラスです

\lstinputlisting[caption=戦略クラス:Tic Tac Toe,label=prog7]{../src/p7.src}

(以下は、上記の//minimax、の部分に納める関数です)

\lstinputlisting[caption=戦略クラス-minimax法:Tic Tac Toe,label=prog8]{../src/p7_1.src}

このMiniMax法は、最後(勝敗がつく)まで読み切るので、決して「負けることはありません」

人間が、最善の手を打った場合には「引き分け」になります

盤面が3✕3と、小さいのでMiniMaxで最後まで読み切ることができるわけですが、
盤面が大きくなると、「アルファ・ベータ・刈り」が必要になります。

\section{最終的な完成形}

最終的なソースは、tictactoeフォルダの中に納めています。

\begin{quotation}
\% ls -l tictactoe\\
total 56\\
-rw-r--r--  1 mat  staff  3780  2 20 14:56 Board.java\\
-rw-r--r--  1 mat  staff   691  2 20 14:01 Constants.java\\
-rw-r--r--  1 mat  staff  1461  2 20 14:44 Game.java\\
-rw-r--r--  1 mat  staff   970  2 20 14:11 Player.java\\
-rw-r--r--  1 mat  staff   363  2 20 14:24 Stone.java\\
-rw-r--r--  1 mat  staff  2657  2 20 14:59 Strategy.java\\
-rw-r--r--  1 mat  staff   350  2 20 14:40 TicTacToe.java\\

\end{quotation}


%\lstinputlisting[caption=完成形:Tic Tac Toe,label=prog9]{../src/TicTacToe.java}


%\section*{謝辞}
%\addcontentsline{toc}{chapter}{謝辞}
%
\begin{thebibliography}{99}
  \bibitem{1}
\end{thebibliography}
%
% END DOCUMENT
\end{document}
%
